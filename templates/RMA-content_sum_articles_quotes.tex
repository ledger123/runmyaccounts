\renewcommand{\arraystretch}{1.3}
 
\ifthenelse{\equal{<%tax%>}{}}{

\parbox{\textwidth}{

\begin{tabularx}{\textwidth}{@{}Xr@{}}

\hline

 <%if quototal%>
 \textbf{\AmountType \ \ <%currency%>}& \textbf{<%quototal%>}\\
 <%end quototal%>

\hline
& \footnotesize{<%company%> ist nicht MWST-pflichtig.}\\

 \end{tabularx}
 }

 }{

\ifthenelse{\equal{<%taxincluded%>}{1}}{

\parbox{\textwidth}{

\begin{tabularx}{\textwidth}{@{}Xr@{}}

\hline

<%if quototal%>
\textbf{\AmountType \ \ <%currency%>\ifthenelse{\equal{<%tax%>}{}}{} { inkl. MWST}}& \textbf{<%quototal%>}\\
<%end quototal%>

\hline

\end{tabularx}

\vspace{0.2cm}

\footnotesize 

\begin{flushright}
\begin{tabular}{@{}lrrr@{}}
\textbf{MWST-Satz} & \textbf{Netto} &  \textbf{MWST} & \textbf{Total}\\
\hline
<%foreach tax%>
<%taxrate%> \%&<%taxbase%>  &<%tax%> & <%taxbaseinclusive%>\\
<%end tax%>
\end{tabular}

\end{flushright}
}

}{

\parbox{\textwidth}{
\begin{tabularx}{\textwidth}{@{}Xr@{}}

\hline

Zwischentotal <%currency%>& <%subtotal%> \\

<%foreach tax%>
MWST <%taxrate%> \% auf <%taxbase%> & <%tax%> \\
<%end tax%>

\hline

<%if quototal%>
\textbf{\AmountType \ \ <%currency%>}& \textbf{<%quototal%>}
<%end quototal%>

\end{tabularx}

\hline
}
}
}

 
 \renewcommand{\arraystretch}{1}