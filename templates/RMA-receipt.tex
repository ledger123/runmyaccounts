
%%%%%%%%%%%%%  TITEL  %%%%%%%%%%%%%
% Art des Templates: Quittung / Debitorenzahlung / receipt.tex
% Mandant: 
% Erstellungsdatum: 
% Version: 
% Ersteller: 
% � by Run my Accounts AG


%statische Elemente
\documentclass{scrartcl}
\usepackage[latin1]{inputenc}
\usepackage[ngerman]{babel}
\usepackage[T1]{fontenc}
\usepackage{tabularx}
\setlength{\parindent}{0pt}
\usepackage{graphicx}
\usepackage{ifthen}
\usepackage{mathptmx}
\font\ocr=ocrb10
\usepackage[absolute]{textpos}
\usepackage{colortbl}
\usepackage{color}
\usepackage{substr}


%dynamische Elemente
%Abst�nde Rand (Anpassung auf die PDF Hintergrund-Vorlage!)
\usepackage[a4paper,top=6cm,bottom=1cm,left=1.8cm,right=1.8cm]{geometry} 

%Schrift
\usepackage[scaled=.90]{helvet} 
\renewcommand{\familydefault}{\sfdefault}

%Tabellendefinitionen
\newcolumntype{L}[1]{>{\raggedright\let\newline\\\arraybackslash\hspace{0pt}}p{#1}} % linksb�ndig mit Breitenangabe
\newcolumntype{C}[1]{>{\centering\let\newline\\\arraybackslash\hspace{0pt}}p{#1}} % zentriert mit Breitenangabe
\newcolumntype{R}[1]{>{\raggedleft\let\newline\\\arraybackslash\hspace{0pt}}p{#1} }% rechtsb�ndig mit Breitenangabe 

%Hintergrund f�r erste Standard-Seite
\usepackage{eso-pic}
\newcommand\BackgroundPic{
\put(0,0){
\parbox[b][\paperheight]{\paperwidth}{%
\vfill
\includegraphics{../<%templates%>/hintergrund.pdf} 
}}}

%Hintergrund f�r Einzahlungsschein (ohne Footer)
\usepackage{eso-pic}
\newcommand\BackgroundPicEZ{
\put(0,0){
\parbox[b][\paperheight]{\paperwidth}{%
\vfill
\includegraphics{../<%templates%>/hintergrundez.pdf} 
}}}

\begin{document}
\AddToShipoutPicture{\BackgroundPic} % MUSS EINKOMMENTIERT WERDEN WENN HINTERGRUND
\shorthandoff{"}
\pagestyle{empty}

% Setzt die generelle Position der Seite. Der bedruckbare bereich kann horizontal / vertikal verschoben werden.
\textblockorigin{0.00cm}{0.00cm} % + nach links / + nach unten

%Schriftwahl
\normalfont

\parbox[t]{12cm}{

Zahlungsempf�nger:\\
\\
<%company%>\\
\BeforeSubString{ --}{<%address%>}\\
\BehindSubString{-- }{<%address%>}\\}

\vspace{1cm}

\textbf{\large{Quittung Nr. <%source%>}}


\vspace{0.5cm}

�ber den Betrag von: <%currency%> <%amount%> {\hfill Bezahlt am: <%datepaid%> }


\vspace{0.5cm}

Bezahlt von:\\
\ifthenelse{\equal{<%typeofcontact%>}{company}}{

%Firmenanschrift

<%name%>

\ifthenelse{\equal{<%contact%>}{}}{}{<%salutation%> <%firstname%> <%lastname%>}

}{

%Privatanschrift

<%salutation%>

<%firstname%> <%lastname%>

} 

\ifthenelse{\equal{address1}{}}{}{<%address1%>}

\ifthenelse{\equal{<%address2%>}{}}{}{<%address2%>}

<%zipcode%> <%city%>

\ifthenelse{\equal{<%country%>}{}}{}{<%country%>}

\vspace{0.5cm}

\ifthenelse{\equal{<%memo%>}{}}{}{<%memo%>

\vspace{0.5cm}
}

\textbf{\large{Bezahlte Rechnung(en):}}

\vspace{0.5cm}

\begin{tabularx}{\textwidth}{@{}lXrr@{}}
\textbf{Rechnungsnummer:} & \textbf{Rechnungsdatum:}
  & \textbf{Betrag <%currency%>} & \textbf{Bezahlt <%currency%>} \\
<%foreach invnumber%>
<%invnumber%> & <%invdate%>  & <%due%> & <%paid%> \\
<%end invnumber%>
\end{tabularx}

%%%%%%%%%%%%%  FOOTER  %%%%%%%%%%%%%

%Die Seite wird aufgef�llt - der Footer ist ganz unten.
\vspace*{\fill}

\begin{footnotesize}
\hline

\vspace{0.4cm}

\parbox[t]{.34\textwidth}{
<%company%>\\
\BeforeSubString{ --}{<%address%>}\\
\BehindSubString{-- }{<%address%>}\\
\ifthenelse{\equal{<%2201_taxnumber%>}{}}{}{MWSt Nr.: <%2201_taxnumber%>}
}
\parbox[t]{.33\textwidth}{
\ifthenelse{\equal{<%tel%>}{}}{}{Tel.: <%tel%>\\}
\ifthenelse{\equal{<%fax%>}{}}{}{Fax: <%fax%>\\}
\ifthenelse{\equal{<%companyemail%>}{}}{}{E-Mail: <%companyemail%>\\}
\ifthenelse{\equal{<%companywebsite%>}{}}{}{Web: <%companywebsite%>\\}
}
\parbox[t]{.33\textwidth}{
\ifthenelse{\equal{<%bankname%>}{}}{}{Bank: <%bankname%>\\}
\ifthenelse{\equal{<%iban%>}{}}{}{IBAN: <%iban%>\\}
\ifthenelse{\equal{<%bic%>}{}}{}{BIC: <%bic%>\\}
}

\end{footnotesize}

\end{document}