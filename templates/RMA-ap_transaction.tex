%%%%%%%%%%%%%  TITLE  %%%%%%%%%%%%%

% Typ: Kreditorenbuchung

\newcommand{\AmountType}{Rechnungsbetrag}

%%%%%%%%%%%%%  DEFINITIONS  %%%%%%%%%%%%%

<%include definitions_packages.tex%>

<%include content_language.tex%>

<%include definitions_language.tex%>

%%%%%%%%%%%%% GENERAL SETTINGS FOR THE TEMPLATE %%%%%%%%%%%%

%%% PAGE LAYOUT %%%
\usepackage[a4paper,top=5cm,bottom=3.1cm,left=1.8cm,right=1.8cm]{geometry} 
\textblockorigin{0.00cm}{0.00cm} % - nach links / + nach unten
\newcommand{\PageNumber}{on}
\newcommand{\PrintOnPaper}{off}
\newcommand{\Greetings}{on}
\newcommand{\Deliveraddress}{on}
\newcommand{\Footer}{on}
\newcommand{\StandardBackground}{../<%templates%>/background_with_footer.pdf}
\newcommand{\BackgroundESRorange}{../<%templates%>/background_ESR.pdf}
\newcommand{\BackgroundESRwhite}{../<%templates%>/background_without_footer.pdf}

%%% PLACE OF ADDRESS %%%
\newcommand{\Addressblock}{right}
%\newcommand{\Addressblock}{left}
%\newcommand{\Addressblock}{pingenleft}

%Tabellendefinitionen
\newcolumntype{L}[1]{>{\raggedright\let\newline\\\arraybackslash\hspace{0pt}}p{#1}} % linksb�ndig mit Breitenangabe
\newcolumntype{C}[1]{>{\centering\let\newline\\\arraybackslash\hspace{0pt}}p{#1}} % zentriert mit Breitenangabe
\newcolumntype{R}[1]{>{\raggedleft\let\newline\\\arraybackslash\hspace{0pt}}p{#1} }% rechtsb�ndig mit Breitenangabe 
\newcolumntype{Z}{>{\raggedright\let\newline\\\arraybackslash\hspace{0pt}}X}

<%include content_footer.tex%>

<%include definitions_settings.tex%>

\pagestyle{myfoot}

%%%%%%%%%%%%%  CONTENT  %%%%%%%%%%%%%

\vspace{1.5cm}

\centerline{\textbf{K R E D I T O R E N B U C H U N G S B E L E G}}

\vspace{1.5cm}

\parbox[t]{.5\textwidth}{
<%name%>

<%address1%>

<%address2%>

<%city%>
<%if state>
\hspace{-0.1cm}, <%state%>
<%end state%> <%zipcode%>

<%country%>

\vspace{0.3cm}

<%if contact%>
<%contact%>
\vspace{0.2cm}
<%end contact%>

<%if vendorphone%>
Tel: <%vendorphone%>
<%end vendorphone%>

<%if vendorfax%>
Fax: <%vendorfax%>
<%end vendorfax%>

<%email%>

<%if vendortaxnumber%>
MWST Nr.: <%vendortaxnumber%>
<%end vendortaxnumber%>
}
\hfill
\begin{tabular}[t]{ll}
  \textbf{Beleg Nr.:} & <%invnumber%> \\
  \textbf{Datum:} & <%invdate%> \\
  \textbf{F�llgikeitsdatum:} & <%duedate%> \\
  <%if ponumber%>
    \textbf{PO-Nummer:} & <%ponumber%> \\
  <%end ponumber%>
  <%if ordnumber%>
    \textbf{Bestell-Nummer:} & <%ordnumber%> \\
  <%end ordnumber%>
  \textbf{Mitarbeiter:} & <%employee%> \\
\end{tabular}

\vspace{1.5cm}

\renewcommand{\arraystretch}{1.3} %Abstand zwischen den Zeilen

\begin{tabularx}{\textwidth}{@{}L{1.9cm}L{4.3cm}L{5.07cm}L{1.3cm}R{3.15cm}@{}}

\hline
 \textbf{Konto Nr.} & \textbf{Konto} & \textbf{Beschreibung} &\textbf{Projekt} & \textbf{Betrag <%currency%>} \\
\hline
\end{tabularx}
\hline

<%foreach amount%>
\begin{tabularx}{\textwidth}{@{}L{1.9cm}L{4.3cm}L{5.07cm}L{1.3cm}R{3.15cm}@{}}
<%accno%> & <%account%> & <%description%> & <%projectnumber%> &<%amount%>\\
\end{tabularx}
\par
\begingroup
\rightskip=7.4cm % Parameter anpassen
\noindent <%itemnotes%>
\par
\endgroup
<%end amount%>

<%include content_sum_bookings.tex%>

\vspace{1cm}

\ifthenelse{\equal{<%notes%>}{}}{}{
<%notes%>
\vspace{1cm}
}


<%if paid_1%>
\begin{tabular}{@{}llllr@{}}
  \multicolumn{5}{c}{\textbf{Zahlungen}} \\
  \hline
  \textbf{Datum} & & \textbf{Beleg} & \textbf{Notiz} & \textbf{Betrag <%currency%>} \\
<%end paid_1%>
<%foreach payment%>
  <%paymentdate%> & <%paymentaccount%> & <%paymentsource%> & <%paymentmemo%> & <%payment%> \\
<%end payment%>
<%if paid_1%>
\end{tabular}
<%end paid_1%>

%%%%%%%%%%%%%  END  %%%%%%%%%%%%%
 
\end{document}