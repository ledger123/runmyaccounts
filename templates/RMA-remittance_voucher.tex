
%%%%%%%%%%%%%  TITEL  %%%%%%%%%%%%%
% Art des Templates: Einzahlungsschein remittance_voucher.tex
% Mandant: 
% Erstellungsdatum: 
% Version: 
% Ersteller: 
% � by Run my Accounts AG


%statische Elemente
\documentclass{scrartcl}
\usepackage[latin1]{inputenc}
\usepackage[ngerman]{babel}
\usepackage[T1]{fontenc}
\usepackage{tabularx}
\setlength{\parindent}{0pt}
\usepackage{graphicx}
\usepackage{ifthen}
\usepackage{mathptmx}
\font\ocr=ocrb10
\usepackage[absolute]{textpos}
\usepackage{colortbl}
\usepackage{color}
\usepackage{substr}


%dynamische Elemente
%Abst�nde Rand (Anpassung auf die PDF Hintergrund-Vorlage!)
\usepackage[a4paper,top=6cm,bottom=1cm,left=1.8cm,right=1.8cm]{geometry} 

%Schrift
\usepackage[scaled=.90]{helvet} 
\renewcommand{\familydefault}{\sfdefault}

%Tabellendefinitionen
\newcolumntype{L}[1]{>{\raggedright\let\newline\\\arraybackslash\hspace{0pt}}p{#1}} % linksb�ndig mit Breitenangabe
\newcolumntype{C}[1]{>{\centering\let\newline\\\arraybackslash\hspace{0pt}}p{#1}} % zentriert mit Breitenangabe
\newcolumntype{R}[1]{>{\raggedleft\let\newline\\\arraybackslash\hspace{0pt}}p{#1} }% rechtsb�ndig mit Breitenangabe 

%Hintergrund f�r erste Standard-Seite
\usepackage{eso-pic}
\newcommand\BackgroundPic{
\put(0,0){
\parbox[b][\paperheight]{\paperwidth}{%
\vfill
\includegraphics{../<%templates%>/hintergrund.pdf} 
}}}

%Hintergrund f�r Einzahlungsschein (ohne Footer)
\usepackage{eso-pic}
\newcommand\BackgroundPicEZ{
\put(0,0){
\parbox[b][\paperheight]{\paperwidth}{%
\vfill
\includegraphics{../<%templates%>/hintergrundez.pdf} 
}}}

\begin{document}
\AddToShipoutPicture{\BackgroundPic} % MUSS EINKOMMENTIERT WERDEN WENN HINTERGRUND
\shorthandoff{"}
\pagestyle{empty}

% Setzt die generelle Position der Seite. Der bedruckbare bereich kann horizontal / vertikal verschoben werden.
\textblockorigin{0.00cm}{0.00cm} % + nach links / + nach unten

%Schriftwahl
\normalfont

%%%%%%%%%%%%%  EZ KOPF  %%%%%%%%%%%%%

\textbf{\large{Rechnung Nr. <%invnumber%>}}

\vspace{0.5cm}

\begin{tabular}{@{}lll}
\textbf{Rechnungsdatum:} & & <%invdate%>\\
\textbf{Rechnungsbetrag:} & & <%currency%> <%invtotal%>\\
\textbf{Zahlungsbedingungen:} & & <%terms%> Tage\\
\textbf{F�lligkeitsdatum:} & & <%duedate%>\\

\end{tabular}

%%%%%%%%%%%%%  EINZAHLUNGSSCHEIN  %%%%%%%%%%%%%

\ifthenelse{\equal{<%bankname%>}{POSTFINANCE}}{

%LINKS POSTFINANCE
%Einzahlung f�r
\begin{textblock*}{5.8cm}(0.6cm,19.6cm)
\begin{flushleft}
<%company%>

<%address%>
\end{flushleft}
\end{textblock*}

}{

%LINKS BANK
%Einzahlung f�r
\begin{textblock*}{5.8cm}(0.6cm,19.6cm)
\begin{flushleft}
<%bankname%>

<%bankzipcode%> <%bankcity%>
\end{flushleft}
\end{textblock*}
 
%Zugunsten von
\begin{textblock*}{5.8cm}(0.6cm,21.1cm)
\begin{flushleft}
<%company%>

\BeforeSubString{ --}{<%address%>}

\BehindSubString{-- }{<%address%>}

\end{flushleft}
\end{textblock*}

}

%Konto
\begin{textblock*}{3cm}(2.8cm,23.1cm)
\begin{flushleft}
\ocr{<%membernumber%>}

\end{flushleft}
\end{textblock*}

%Betrag Franken
\begin{textblock*}{3.7cm}(0.4cm,23.8cm)
\begin{flushright}
\ocr{<%integer_out_amount%>}

\end{flushright}
\end{textblock*}

%Betrag Rappen
\begin{textblock*}{1cm}(5.1cm,23.8cm)
\begin{flushleft}
\ocr{<%out_decimal%>}

\end{flushleft}
\end{textblock*}

%Einbezahlt von
\begin{textblock*}{5.5cm}(0.6cm,24.8cm)
\begin{flushleft}
\ocr{<%dcn group=5right%>}
\normalfont 

<%name%>

<%address1%>

<%zipcode%> <%city%>

\end{flushleft}
\end{textblock*}

\ifthenelse{\equal{<%bankname%>}{POSTFINANCE}}{

%MITTE POSTFINANCE
%Einzahlung f�r
\begin{textblock*}{6cm}(6.3cm,19.6cm)
\begin{flushleft}
<%company%>

<%address%>
\end{flushleft}
\end{textblock*}

}{

%MITTE BANK
%Einzahlung f�r 
\begin{textblock*}{6cm}(6.3cm,19.6cm)
\begin{flushleft}
<%bankname%>

<%bankzipcode%> <%bankcity%>
\end{flushleft}
\end{textblock*}
 
%Zugunsten von
\begin{textblock*}{6cm}(6.3cm,21.1cm)
\begin{flushleft}
<%company%>

\BeforeSubString{ --}{<%address%>}

\BehindSubString{-- }{<%address%>}
\end{flushleft}
\end{textblock*}

}

%Konto
\begin{textblock*}{6cm}(8.9cm,23.1cm)
\begin{flushleft}
\ocr{<%membernumber%>}

\end{flushleft}
\end{textblock*}

%Betrag Franken
\begin{textblock*}{3.9cm}(6.3cm,23.8cm)
\begin{flushright}
\ocr{<%integer_out_amount%>}
\end{flushright}
\end{textblock*}

%Betrag Rappen
\begin{textblock*}{1cm}(11.15cm,23.8cm)
\begin{flushleft}
\ocr{<%out_decimal%>}

\end{flushleft}
\end{textblock*}

%Kodierzeile
\begin{textblock*}{16cm}(6.85cm,27.2cm)
\begin{flushleft}
\ocr{<%rvc%>><%dcn%>+ <%bankstate%>>}
\end{flushleft}
\end{textblock*}

%RECHTS
%Referenz-Nr.
\begin{textblock*}{8.3cm}(12.38cm,22.1cm)
\begin{flushleft}
\ocr{<%dcn group=5right%>}

\end{flushleft}
\end{textblock*}

%Name
\begin{textblock*}{6.5cm}(12.38cm,24.4cm)
\begin{flushleft}
<%name%>

<%address1%>

<%zipcode%> <%city%>


\end{flushleft}
\end{textblock*}

%DIESE BEIDEN KLAMMERN WERDEN GEBRAUCHT, WENN DIE EINF�GUNG DES EZ AUTOMATISCH ERFOLGT
}
}
}
}

\end{document}